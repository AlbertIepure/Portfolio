%% ++++++++++++++++++++++++++++++++++++++++++++++++++++++++++++
%% Thesenpapier, als Handout
%% ++++++++++++++++++++++++++++++++++++++++++++++++++++++++++++
%
%  Gerüst:
%  * Version 0.10
%  * Dipl.-Ing. Florian Evers, florian.evers@tu-ilmenau.de
%  * Fachgebiet Kommunikationsnetze, TU Ilmenau
%
%  Für Hauptseminare, Studienarbeiten, Diplomarbeiten
%
%  Autor           : Max Mustermann
%  Letzte Änderung : 31.12.2004
%

% Headerfeld, Typ des Dokumentes, einzubindende Packages.
% Hier bei Bedarf Änderungen vornehmen.
\documentclass
[   12pt,           % Bezug: 12-Punkt Schriftgröße
    DIV15,          % Randaufteilung, siehe Dokumentation ``KOMA''-Script
    a4paper         % Seitenformat A4
]   {scrreprt}      % Verwende ``Report'' aus KOMA-Skript-Paket

\usepackage[active]{srcltx}
%\usepackage[activate=normal]{pdfcprot} % Optischer Randausgleich -> pdflatex!
\usepackage{ifthen}
\usepackage{ngerman}
\usepackage[latin1]{inputenc}
\usepackage[T1]{fontenc}
\usepackage[T1]{url}
\usepackage{ae}
\usepackage[final]{graphicx}
\usepackage{scrpage2}

% Seitenlayout festlegen. Hier nichts ändern!
\pagestyle{scrplain}
\ihead[]{}
\ohead[]{}
\chead[]{}
\ifoot[]{}
\ofoot[]{}
\cfoot[]{}

% Hier in die zweite geschweifte Klammer jeweils
% die persönlichen Daten eintragen:
\newcommand{\artderausarbeitung}{Diplomarbeit}
\newcommand{\namedesautors}{Max Mustermann}

\begin{document}
%% ++++++++++++++++++++++++++++++++++++++++++++++++++++++++++++
%% Thesen zur Ausarbeitung. Für Diplomarbeiten
%% ++++++++++++++++++++++++++++++++++++++++++++++++++++++++++++
%
%  Gerüst:
%  * Version 0.11
%  * Dipl.-Ing. Karsten Renhak, karsten.renhak@tu-ilmenau.de
%  * Fachgebiet Kommunikationsnetze, TU Ilmenau
%
%  Für Hauptseminare, Studienarbeiten, Diplomarbeiten
%
%  Autor           : Max Mustermann
%  Letzte Änderung : 31.12.2011
%

\chapter*{Thesen zur \artderausarbeitung}
\addcontentsline{toc}{chapter}{Thesen zur \artderausarbeitung}
\ihead[]{Thesen zur \artderausarbeitung}

\begin{enumerate}
\item Mit \LaTeX\ gesetzte Dokumente sehen überall
      gleich aus. Sie werden ähnlich wie HTML in Klartext
      geschrieben und anschließend mit Hilfe eines Konverters in
      Postscript- oder PDF"=Dateien gewandelt.
\item \LaTeX\ gibt es für alle wichtigen Betriebssysteme.
\item Die Benutzung einer integrierten Entwicklungsumgebung,
      beispielsweise {\ttfamily Kile} oder {\ttfamily TeXnicCenter},
      wird empfohlen.
\item Dieses Dokument ist Formatvorlage und Einstiegshilfe
      zugleich. Einfach den Text durch die eigene Ausarbeitung
      ersetzen.
\end{enumerate}

% Etwas Platz schaffen:
\section*{}

Ilmenau, den 31.\,12.\,2011\hfill \namedesautors

\end{document}


